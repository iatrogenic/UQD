\documentclass[reqno,a4paper,12pt]{amsart}%{article}%
\usepackage{amssymb,amsmath,amscd,amstext,amsthm,amsfonts}
\usepackage[]{graphicx}
\usepackage{epstopdf}
\usepackage{setspace} 
\usepackage{subfig}
\usepackage{color}
\usepackage[mathscr]{eucal}
\usepackage[draft]{changes}
%\doublespacing
\usepackage[ansinew]{inputenc} 
\usepackage{hyperref}
\usepackage{color}
%\usepackage{standalone}
\usepackage{hyperref}
\usepackage{subfiles}

\usepackage{quoting}

%\usepackage{mathtools}
\usepackage{nicefrac}
%\usepackage{xfrac}
\usepackage{physics}
\usepackage{enumerate}
\usepackage{bbm}


%\usepackage[pagebackref]{hyperref}
%\renewcommand*{\backref}[1]{Cited at page [#1]}

%%%%%%%%%%%%%%%%%%%%%%%%%%%%%%%%%%%%%%%%%%%%%%%%%%%%%
%\addtolength{\evensidemargin}{-15mm}
%\addtolength{\oddsidemargin}{-15mm}
%\addtolength{\textwidth}{30mm}
%\addtolength{\textheight}{20mm}
%\addtolength{\topmargin}{-10mm}
%%%%%%%%%%%%%%%%%%%%%%%%%%%%%%%%%%%%%%%%%%%%%%%%%%%%
\renewcommand{\theequation}{\thesection.\arabic{equation}}
\numberwithin{equation}{section}
%%%%%%%%%%%%%%%%%%%%%%%%%%%%%%%%%%%%%%%%%%%%%%%%%%%%%
\newtheorem{thm}{Theorem}[section]
%\newtheorem{lem}[thm]{Lemma}%[section]
%\newtheorem{prop}[thm]{Proposition}%[section]
%\newtheorem{cor}[thm]{Corollary}%[section]
%%\newtheorem{Jac}{Jacobi}

%%%%%%%%%%%%%%%%%%%%%%%%%%%%%%%%%%%%%%%%%%%%%%%%%%%%%
%\newtheorem{theorem}{Theorem}[section]
%\newtheorem{proposition}[theorem]{Proposition}
%\newtheorem{conjecture}[theorem]{Conjecture}
%\newtheorem{corollary}[theorem]{Corollary}
%\newtheorem{lemma}[theorem]{Lemma}
%\newtheorem{estimate}[theorem]{Estimate}
%\newtheorem{assumption}[theorem]{Assumption}
%{\theoremstyle{definition}%{plain}
	%{%\theorembodyfont{\normalfont\rmfamily}
		%%\newthbeorem{definition}[theorem]{Definition}
		%\newtheorem{remark}[theorem]{Remark}
		%\newtheorem{example}[theorem]{Example}
		%\newtheorem{exercise}[section]{Exercise}
		%\newtheorem{defn}[theorem]{Definition}
		%}}

\newtheorem{theorem}{Theorem}
\newtheorem{proposition}[theorem]{Proposition}
\newtheorem{lemma}[theorem]{Lemma}
\theoremstyle{definition}
\newtheorem{remark}[theorem]{Remark}
\newtheorem{definition}[theorem]{Definition}
\newtheorem{example}{Example}

%%%%%%%%%%%%%%%%%%%%%%%%%%%%%%%%%%%%%%%%%%%%%%%%%%%%



\newcommand{\cal}{\mathcal}

%\newcommand{\A}{{\cal A}}
%\newcommand{\BB}{{\cal B}}
%\newcommand{\CC}{{\cal C}}
%\newcommand{\CCC}{{\cal C}}
%\newcommand{\DD}{{\cal D}}
%\newcommand{\EE}{{\cal E}}
%\newcommand{\FF}{{\cal F}}
%\newcommand{\GG}{{\cal G}}
%\newcommand{\HH}{{\cal H}}
%\newcommand{\II}{{\cal I}}
%\newcommand{\JJ}{{\cal J}}
%\newcommand{\KK}{{\cal K}}
%\newcommand{\LL}{{\cal L}}
%\newcommand{\MM}{{\cal M}}
%\newcommand{\NN}{{\cal N}}
%\newcommand{\OO}{{\cal O}}
%\newcommand{\PP}{{\cal P}}
%\newcommand{\QQ}{{\cal Q}}
%\newcommand{\QQQ}{{\cal Q}}
%\newcommand{\RR}{{\cal R}}
%\newcommand{\SSS}{{\cal S}}
%\newcommand{\TT}{{\cal T}}
%\newcommand{\TTT}{{\cal T}}
%\newcommand{\UU}{{\cal U}}
%\newcommand{\VV}{{\cal V}}
%\newcommand{\WW}{{\cal W}}
%\newcommand{\XX}{{\cal X}}
%\newcommand{\YY}{{\cal Y}}
%\newcommand{\ZZ}{{\cal Z}}
%\newcommand{\SRB}{\mathscr{E}}
%\newcommand{\Mix}{{\rm Mix}}

\newcommand{\fa}{{\mathfrak a}}
\newcommand{\fg}{{\mathfrak g}}
\newcommand{\fu}{{\mathfrak u}}
\newcommand{\fD}{{\mathfrak D}}
\newcommand{\fF}{{\mathfrak F}}
\newcommand{\fG}{{\mathfrak G}}
\newcommand{\fN}{{\mathfrak N}}
\newcommand{\fR}{{\mathfrak R}}
\newcommand{\fU}{{\mathfrak U}}
\newcommand{\fX}{{\mathfrak X}}


\newcommand{\Aa}{{\mathbb{A}}}
\newcommand{\Bb}{{\mathbb{B}}}
\newcommand{\Cc}{{\mathbb{C}}}
\newcommand{\Dd}{{\mathbb{D}}}
\newcommand{\Ee}{{\mathbb{E}}}
\newcommand{\Ff}{{\mathbb{F}}}
\newcommand{\Gg}{{\mathbb{G}}}
\newcommand{\Hh}{{\mathbb{H}}}
\newcommand{\Ii}{{\mathbb{I}}}
\newcommand{\Jj}{{\mathbb{J}}}
\newcommand{\Kk}{{\mathbb{K}}}
\newcommand{\Ll}{{\mathbb{L}}}
\newcommand{\Mm}{{\mathbb{M}}}
\newcommand{\Nn}{{\mathbb{N}}}
\newcommand{\Oo}{{\mathbb{O}}}
\newcommand{\Pp}{{\mathbb{P}}}
\newcommand{\Qq}{{\mathbb{Q}}}
\newcommand{\Rr}{{\mathbb{R}}}
\newcommand{\Ss}{{\mathbb{S}}}
\newcommand{\Tt}{{\mathbb{T}}}
\newcommand{\Uu}{{\mathbb{U}}}
\newcommand{\Vv}{{\mathbb{V}}}
\newcommand{\Ww}{{\mathbb{W}}}
\newcommand{\Xx}{{\mathbb{X}}}
\newcommand{\Yy}{{\mathbb{Y}}}
\newcommand{\Zz}{{\mathbb{Z}}}


\newcommand{\Q}{{\mathbb{Q}}}
\newcommand{\Z}{{\mathbb{Z}}}

\DeclareMathOperator{\sint}{int}
%%%%%%%%%%%%%%%%%%%%%%%%%%%%%%%%%%%%%%%%%%%%%

\def\H{{\mathfrak H}}
\def\Iso{\operatorname{Iso}}
\def\e{\mathrm{e}}
%\def\i{\mathrm{i}}
\def\cc{\operatorname{cc}}
\def\diag{\operatorname{diag}}
\def\dist{\operatorname{dist}}
\def\error{\operatorname{error}}
%\def\id{\operatorname{id}}
\def\j{\operatorname{j{}}}
\def\C{\operatorname{C{}}}
\def\G{\operatorname{G{}}}
\def\L{\operatorname{L{}}}
\def\M{\operatorname{M{}}}
\def\Mat{\operatorname{Mat}}
\def\GL{\operatorname{GL}}
\def\Op{\operatorname{Op}}
\def\sOp{\widetilde{\operatorname{Op}}}
\def\PGL{\operatorname{PGL}}
\def\Res{\operatorname{Res}}
\def\S{\operatorname{S{}}}
\def\Sp{\operatorname{Sp}}
\def\Sw{{\mathcal S}}
\def\SL{\operatorname{SL}}
\def\sl{\operatorname{sl}}
\def\SO{\operatorname{SO}}
\def\PSL{\operatorname{PSL}}
\def\O{\operatorname{O{}}}
\def\T{\operatorname{T{}}}
\def\tr{\operatorname{tr}}
\def\sgn{\operatorname{sgn}}
\def\supp{\operatorname{supp}}
\def\meas{\operatorname{meas}}
\def\Leb{\operatorname{Leb}}
\def\Var{\operatorname{Var}}
\def\Vol{\operatorname{Vol}}
\def\Area{\operatorname{Area}}
\def\ord{\operatorname{ord}}
\def\Prob{\operatorname{Prob}}
\def\Ad{\operatorname{Ad}}
\def\ad{\operatorname{ad}}

\def\Per{\operatorname{Per}}

\def\GamG{\Gamma\backslash G}
\def\SLZ{\SL(2,\Zz)}
\def\SLR{\SL_2(\Rr)}
\def\SLC{\SL_2(\Cc)}
\def\SOR{\SO(2)}

\def\Re{\operatorname{Re}}
\def\Im{\operatorname{Im}}

%%%%% Pedro's Macro
\newcommand{\Mod}[1]{\left\vert{#1}\right\vert}
\newcommand{\dem}{ \par\medbreak\noindent{\bf
		Proof. }\enspace} 
\newcommand{\cqd}{\hfill
	$\sqcup\!\!\!\!\sqcap\bigskip$}
\newcommand{\nrm}[1]{\left\|#1\right\|}


%\def\trans{\,^\top\!}

%\DeclareMathOperator{\GL}{GL}
%\DeclareMathOperator{\SL}{SL}
\DeclareMathOperator{\vspan}{span}

\newcommand {\CC}{\mathbb{C}}
\newcommand {\RR}{\mathbb{R}}
\newcommand {\FF}{\mathbb{F}}
\newcommand {\NN}{\mathbb{N}}
\newcommand {\QQ}{\mathbb{Q}}
\newcommand {\EE}{\mathbb{E}}
\newcommand {\PP}{\mathbb{P}}


%%%%%%%%%%%%%%%%%%%%%%%%%%%%%%%%%%%%%%%%%%%%%%%%%%%%%%%%%%%%%%%%%%%%%

%\newcommand{\abs}[1]{\left| #1\right|}
%\newcommand{\norm}[1]{\left\| #1\right\|}
%\newcommand{\pde}[2]{\frac{\partial #1}{\partial #2}}

%\newcommand{\spec}[1]{\operatorname{spec}\left( #1\right)}
%\newcommand{\tr} {\operatorname{tr}}
%\newcommand{\diag} {\operatorname{diag}}
\newcommand{\trans} {\,^\top\!}
\newcommand{\conj} {\overline}
\newcommand{\para} {\parallel}

\newcommand{\id}  {\operatorname{id}}
\newcommand{\im}  {\operatorname{Im}}
\newcommand{\re}  {\operatorname{Re}}
%\newcommand{\up}  {\uparrow}
%\newcommand{\down}{\downarrow}
%\newcommand{\sgn} {\operatorname{sgn}}
%\newcommand{\const}{\operatorname{cst}}

\newcommand{\Int} {\operatorname{int}}
%\newcommand{\Ext} {\operatorname{Ext}}
%\newcommand{\Span}{\operatorname{span}}
%\newcommand{\perm}{\operatorname{Perm}}
%\newcommand{\std} {\operatorname{std}}
%\newcommand{\fix} {\operatorname{Fix}}
%\newcommand{\cl}  {\operatorname{ cl}}

\newcommand{\rot}{\operatorname{rot}}
\newcommand{\Homeo}{\operatorname{Homeo}} %{\mbox{\rm Homeo\,}}
\newcommand{\Diff} {\operatorname{Diff}}  %{\mbox{\rm Diff\,}}
\newcommand{\vf}   {\operatorname{Vect}} % {\XX}
%\newcommand{\diffo}{\text{Diff}_0\,}
%\newcommand{\vol}  {\operatorname{Diff}_{\text{vol}}\,}
%\newcommand{\Vol}  {\operatorname{vol}}
\newcommand{\Symp} {\operatorname{Symp}}
%\newcommand{\sympo}{\mbox{\rm Symp_0\,}} % {\text{Symp}_0\,}
\newcommand{\Ham}  {\operatorname{Ham}}
\newcommand{\Sym} {\operatorname{Sym}}
\newcommand{\Skew} {\operatorname{Skew}}
%\newcommand{\Sp}   {\operatorname{Sp}}
%\newcommand{\lie}  {\mbox{\rm Lie\,}}

%\newcommand{\GL}   {\operatorname{GL}}
%\newcommand{\SL}   {\operatorname{SL}}
%%\newcommand{\SU}   {\operatorname{SU}}
%\newcommand{\SO}   {\operatorname{SO}}

\newcommand{\DC}   {\operatorname{DC}}
\newcommand{\LC}   {\operatorname{LC}}

\newcommand{\te}[1]{\quad\text{#1}\quad}
%\newcommand{\comment}[1]{}

\newcommand{\Teich}{\mathscr{T}}
\newcommand{\TTeich}{{\rm T}\mathscr{T}}
\newcommand{\Bundle}{\mathscr{B}}
\newcommand{\Afrk}{\mathfrak{A}}
\newcommand{\Bfrk}{\mathfrak{B}}
\newcommand{\Cfrk}{\mathfrak{C}}
\newcommand{\Dfrk}{\mathfrak{D}}
\newcommand{\Ffrk}{\mathfrak{F}}
\newcommand{\Hfrk}{\mathfrak{H}}
\newcommand{\Lfrk}{\mathfrak{L}}
\newcommand{\Ofrk}{\mathfrak{O}}
\newcommand{\Qfrk}{\mathfrak{Q}}
\newcommand{\Rfrk}{\mathfrak{R}}
\newcommand{\Tfrk}{\mathfrak{T}}
\newcommand{\Ufrk}{\mathfrak{U}}
\newcommand{\Ascr}{\mathscr{A}}
\newcommand{\Bscr}{\mathscr{B}}
\newcommand{\Cscr}{\mathscr{C}}
\newcommand{\Escr}{\mathscr{E}}
\newcommand{\Zscr}{\mathscr{Z}}
\newcommand{\Gscr}{\mathscr{G}}
\newcommand{\Nscr}{\mathscr{N}}
\newcommand{\Mscr}{\mathscr{M}}
\newcommand{\Kscr}{\mathscr{K}}
\newcommand{\Iscr}{\mathscr{I}}
\newcommand{\Jscr}{\mathscr{J}}
\newcommand{\Dscr}{\mathscr{D}}
\newcommand{\Fscr}{\mathscr{F}}
\newcommand{\Hscr}{\mathscr{H}}
\newcommand{\Lscr}{\mathscr{L}}
\newcommand{\Oscr}{\mathscr{O}}
\newcommand{\Qscr}{\mathscr{Q}}
\newcommand{\Pscr}{\mathscr{P}}
\newcommand{\Rscr}{\mathscr{R}}
\newcommand{\Sscr}{\mathscr{S}}
\newcommand{\Tscr}{\mathscr{T}}
\newcommand{\Uscr}{\mathscr{U}}
\newcommand{\Vscr}{\mathscr{V}}
\newcommand{\Wscr}{\mathscr{W}}
\newcommand{\Xscr}{\mathscr{X}}
\newcommand{\Grp}{{\rm G}}
\newcommand{\Grass}{\mathscr{G}}
\newcommand{\Sperm}{{\rm S}}
\newcommand{\SJ}{{\rm S}^{sp}}
\newcommand{\Flag}{\Fscr}
\newcommand{\Fliso}{\Fscr^{sp}}
\newcommand{\Grptil}{\widetilde{\rm G}}
\newcommand{\Vfrk}{\mathfrak{V}}
\newcommand{\Xfrk}{\mathfrak{X}}
\newcommand{\Gfrk}{\mathfrak{G}}
\newcommand{\Pfrk}{\mathfrak{P}}
\newcommand{\Mfrk}{\mathfrak{M}}
\newcommand{\Ical}{\mathcal{I}}
\newcommand{\Grad}{{\rm Grad}}
\newcommand{\Sone}{[0,1]}

\newcommand{\emb}[1]{\BB^{#1}}

\newcommand{\ff}{\mathbb{II}}




\begin{document}
	\title[A Sickness Unto Death]{-- Lecture  0 -- \\
	\Large A Sickness Unto Death \\
	\small or, An Enquiry into the Theory of Measure as it Concerns those Processes of a Stochastic Kind
	}
	\author[Underground Research Division]{Underground Research Division
	\\ 
	DO NOT DISTRIBUTE THIS DOCUMENT WITHOUT PERMISSION }
	\date{\today}
	\maketitle
	
	\begin{center}
		\includegraphics[width=5cm, height=5cm]{urd.jpg}
	 \end{center}
	
\pagebreak

\iffalse  
  \section*{Preface}

\textit{This preface is temporary. I just want to quickly sketch a road man to guide our study.}

\smallbreak

A good end goal, for now, would be to prove \textbf{Girsanov's Theorem}. It is an important result and will require a lot of the fundamentals:

\smallbreak

\begin{enumerate}
	\item Probability spaces and Information
	\begin{enumerate}
		\item $\sigma$-algebras
		\begin{enumerate}
			\item Filtrations
		\end{enumerate}
		\item Independence and Conditioning
		\item Adapted Stochastic Processes
		\begin{enumerate}
			\item Martigales
			\item Markov Processes
		\end{enumerate}
	\end{enumerate}
	\item Brownian motion
	\begin{enumerate}
		\item Random Walks
		\item Martingales and Symmetric Random Walks
		\item Distribution of Brownian Motion
		\item Filtrations for Brownian Motion
		\item Martingales and Brownian Motion
	\end{enumerate}
	\item It\^{o} calculus
	\begin{enumerate}
		\item It\^{o} integral (Simple case)
		\begin{enumerate}
			\item Definition, construction and properties
		\end{enumerate}
		\item It\^{o} integral (General case)
		\item It\^{o}-Doeblin formula for Brownian motion
		\item Levy's Theorem
	\end{enumerate}
	\item Black-Scholes-Merton Equation 
	\item Radon-Nikodym 
	\begin{enumerate}
		\item Radon-Nikodym Theorem
		\item Radon-Nikodym derivatives
	\end{enumerate}
	\item Risk-Neutral Measure
	\begin{enumerate}
		\item Girsanov's Theorem
		\item Novikov's condition
		\item Risk-neutral pricing in the Black-Scholes model
	\end{enumerate}
\end{enumerate}
	 
\vfil
\fi


\pagebreak	

\tableofcontents

\section*{Nihil -- $\emptyset$}
	
\begin{quoting}
``It is clear that I can only deliver to you, to each of you, what you are already on the verge of absorbing."  
\end{quoting}

\begin{center}
\par\noindent\rule{200pt}{0.1pt}
\end{center}
		
\section{The Doctrine of Chances} 

The general problem of measure is our starting point. Given $A \subseteq \Omega$ we want to be able to assign a quantity $m(A)$ to this set, which can be said, in some sense, to be its ``measure". Take, for example, $X=\RR$  and $A = [a,b]$, for which we can set $m (A)  = b-a$. Some questions immediately arise: what properties must $m$ obey? and what is its proper domain, i.e. which sets can be said to be \textit{measurable}? The answer to this last question requires the introduction of the elementary structure of measure theory-- the $\sigma$-algebra.
\begin{definition}
	A  \textbf{$\sigma$-algebra over a set $\Omega$} is a set $\mathcal{F} \subseteq \mathcal{P}(\Omega)$ satisfying
	\begin{enumerate}
		\item $\emptyset \in \mathcal{F}$,
		\item $A \in \mathcal{F} \implies \Omega \setminus A \in \mathcal{F}$,
		\item If $A_1, A_2, \ldots \in \mathcal{F}$ then
		\[
		\bigcup_{i \in \NN} A_i \in \mathcal{F}.
		\]
	\end{enumerate}
\end{definition}
	
\begin{definition}
	Let $\mathcal{F}$ be a $\sigma$-algebra over a set $\Omega$. A \textbf{probability measure} $\Pp$ is a  function $\Pp: \mathcal{P}(\mathcal{F}) \rightarrow [0,1] $ such that:
\begin{enumerate}
	\item $\Pp (\Omega) =1,$
	\item  for disjoint $A_1, A_2,... \in \mathcal{F}:$ $$\Pp \Bigg( \bigcup^{\infty}_{n=1} A_n \Bigg) = \sum^{\infty}_{n=1} \Pp (A_n) $$
\end{enumerate}
 The triple $(\Omega, \mathcal{F}, \Pp)$ is called a \textbf{probability space}.
\end{definition}

\begin{definition}
	Let $(\Omega, \mathcal{F}, \Pp)$ be a probability space and $A \in \mathcal{F}$. 
	
	If $\Pp (A)= 1$ we say that $A$ occurs \textbf{almost surely} or \textbf{almost always}.
	
	If $\Pp (A)= 0$ we say that $A$ occurs \textbf{almost never}. 
\end{definition}



				

\begin{definition}
	Let $(\Omega, \mathcal{F}, \Pp)$ be a probability space. A \textbf{random variable} $X$ is function $X: \Omega \rightarrow \Rr $ such that 
	$$ \{\omega \in \Omega \mid X(\omega) \in B\} \in \mathcal{F},$$ for every Borel subset $B$ or $\Rr.$
	
	We will also denote the set $\{\omega \in \Omega \mid X(\omega) \in B\}$ simply by $\{X \in B\}.$
\end{definition}

\begin{definition}
 Let $X$ be a random variable on a probability space  $(\Omega, \mathcal{F}, \Pp)$. The
\textbf{distribution measure of} $X$ is the probability measure $\mu_X$ that assigns to each Borel
subset $B$ of $\Rr$ the \textbf{mass} $$\mu_X(B) = \Pp (\{X \in B\}).$$.
\end{definition}

\subsection{Expected Value}

\begin{center}
	\par\noindent\rule{200pt}{0.1pt}
\end{center}

\


\begin{center}

In data's dark, a guardian sleeps

Expected Value, veiled, mystery deep

Probability's weights, measures guide

Assigning worth, uncertainty's tide

Chancing fate, in shadows cast

Converging sums, destiny forecast

Lo, a shadow of horror is risen 

In Eternity! Unknown, unprolific? 

Self-closd, all-repelling; what Demon 

Hath form'd this abominable void 

This soul-shudd'ring vacuum? 

Some said 

``It is Urizen". But unknown, abstracted 

Brooding secret, the dark power hid.	
\end{center}


\

\begin{center}
	\par\noindent\rule{200pt}{0.1pt}
\end{center}

\

Let $(\Omega, \mathcal{F}, \Pp)$ be a probability space with  a random variable $X$. Then  
$$\Ee X= \sum_{\omega \in \Omega} X(\omega) \Pp (\omega)$$ for the finite case,

$$\Ee X= \sum_{k=1}^\infty X(\omega_k) \Pp (\omega_k),  $$ for the contable case.

The uncontable case is problematic. Consider

$$X^+(\omega)=\max \{X(\omega),0\} \textrm{ and } X^-(\omega)=\max \{-X(\omega),0\} $$

$$\int_\Omega  X^+(\omega) d \Pp (\omega) \textrm{ and} \int_\Omega X^-(\omega) d \Pp (\omega)  $$

and then 

$$ \int_\Omega X(\omega) d \Pp (\omega) = \int_\Omega X^+(\omega) d \Pp (\omega)  -\int_\Omega X^-(\omega) d \Pp (\omega)  $$

 

\begin{theorem}
	Let $X$ and $Y$ be a random variables on a probability space $(\Omega, \mathcal{F}, \Pp)$.
	Then:
	 \begin{enumerate}
		\item 	 if $X$ takes a finite amount of values $x_1, x_2,..., x_n,$ then $$\int_\Omega X(\omega)d \Pp (\omega)= \sum_{k=1}^n x_k \Pp (\{ X= x_k \}) $$
		\item $X^+(\omega)=\max \{X(\omega),0\} < \infty, \ X^-(\omega)=\max \{-X(\omega),0\} < \infty $  $$  \textrm{if and only if } \quad \quad \int_\Omega 	| X(\omega) | d \Pp (\omega)< \infty  $$
		\item  If $X \leq Y$ almost surely, 
		$$  \int_\Omega X(\omega) d \Pp (\omega) \leq \int_\Omega Y(\omega) d \Pp (\omega) \quad \quad  \textrm{(if both integrals exist) }  $$ and the equality is realized if $X=Y$ almost surely	
		\item if $\alpha, \beta \in \Rr$   $$  \int_\Omega (\alpha X(\omega)+ \beta Y(\omega)) d \Pp (\omega)= 	 \alpha\int_\Omega X(\omega) d \Pp (\omega)+ \beta\int_\Omega Y(\omega) d \Pp (\omega)    $$
	\end{enumerate}
\end{theorem}



The \textbf{indicator function} will often be useful: $$ \Ii_A (\omega)= 1, \textrm{ if }  \omega \in A \textrm{ else } \Ii_A (\omega)=0,$$ for any set $A$.

If we want to integrate a ramdom variable $X$ over a subset $A$ of $\Omega$, we may use the \textbf{indicator function} to define 

$$ \int_A X(\omega) d \Pp (\omega) = \int_\Omega \Ii_A X(\omega) d \Pp (\omega) $$

\begin{definition}
	Let $X$ be a random variable on a probability space $(\Omega, \mathcal{F}, \Pp)$. The \textbf{expected value} of $X$ is 
	$$\Ee X = \int_\Omega X(\omega) d \Pp (\omega)$$
\end{definition}

Care must be taken when the integrals diverge.

\begin{theorem}
	Let $X$ and $Y$ be a random variables on a probability space $(\Omega, \mathcal{F}, \Pp)$.
	\begin{enumerate}
		\item 	 if $X$ takes a finite amount of values $x_1, x_2,..., x_n,$ then $$\Ee X= \sum_{k=1}^n x_k \Pp (\{ X= x_k \}) $$
		\item $X^+(\omega)=\max \{X(\omega),0\} < \infty, \ X^-(\omega)=\max \{-X(\omega),0\} < \infty $  $$  \textrm{if and only if } \quad \quad \Ee| X| < \infty  $$
		\item  If $X \leq Y$ almost surely, 
		$$  \Ee X \leq \Ee Y d \Pp (\omega) \quad \quad  \textrm{(if both integrals exist) }  $$ and the equality is realized if $X=Y$ almost surely	
		\item if $\alpha, \beta \in \Rr$   $$   \Ee(\alpha X+ \beta Y) = 	 \alpha \Ee X + \beta \Ee Y    $$
		\item (Jensen's inequality) If $\varphi: \Rr \to \Rr$ is a convex and $\Ee |X| < \infty,$ then $$\varphi(\Ee X)\leq \Ee \varphi(X) $$
	\end{enumerate}
\end{theorem}
 



 
\begin{center}
	\par\noindent\rule{200pt}{0.1pt}
\end{center}	

\

\begin{center}	

Times on times he divided, \& measur'd 

Space by space in his ninefold darkness 

Unseen, unknown: changes appeard 

In his desolate mountains rifted furious 

By the black winds of perturbation

In realms of measure, sets unfold

Borel's shadow, secrets told

Sigma fields converge to bind

Lebesgue's dream, precision aligned

Cantor's dust, infinite unfold

Measurable paths, stories untold

Open closed, boundaries blur

Topology's dance, precision furor


\end{center}

\

\begin{center}
	\par\noindent\rule{200pt}{0.1pt}
\end{center}	
	
\
	

\begin{definition}
	The $\sigma$-algebra of \textbf{Borel subsets of} $\Rr$, denoted by $\mathcal{B}(\Rr)$ is the $\sigma$-algebra generated from the closed real intervals $[a,b].$ \\
	Informally, this is no more than collection of subsets of $\Rr$ that can be contructed from closed intervals using a countable amount of intersections, unions and complements
	``Most" real subsets are Borel subsets. It takes a substancial amount of effort to find contra-examples.
\end{definition}

\begin{definition}
	The Lebesgue measure on $\Rr$ which we denote by $\mathcal{L},$ maps each $B \in \mathcal{B}(\Rr)$ to $[0, \infty]$ (that is, a non-negative real of $\infty$) such that
	\begin{enumerate}
		\item $\mathcal{L}([a,b])= b-a$, when $a\leq b$
		\item if $B_1, B_2, B_3,...$ is a sequence of disjoint sets in $\mathcal{B}(\Rr),$ then $$\mathcal{L}\Bigg(\bigcup_{n=1}^\infty B_n\Bigg)= \sum_{n=1}^\infty $$
	\end{enumerate}
\end{definition}





\begin{theorem}
	Let $f: \Rr \to \Rr $ be a bounded real function, and let $a<b,$ be reals.
	\begin{enumerate}
		\item The Riemann integral of $\int_a^b f(x)dx$ exists if and only if the subset of $[a,b]$ where $f$ is not continuous has Lebesgue measure zero
		\item If  the Riemann integral of $\int_a^b f(x)dx$ exists then $f$ is Borel measurable (in particular, the Lebesgue integral $\int_a^b f(x)d\mathcal{L}(x) x$ exists ) and the integrals are the same.
	\end{enumerate} 
\end{theorem}
  
\begin{definition}
	Given a property, if the set of reals that fail to have is a set with Lebesgue measure zero, we say that the property \textbf{holds almost aeverywhere}. 
\end{definition}
 
As the Riemann and Lebesgue integrals agree (when the Riemann integral exists), we will use simply the term \textit{integral} to refer to either, and abuse the notation when no confuse arises.
$\int_a^b f(x) dx $ will denote the Lebesgue integral (instead of $\int_{[a,b]}f(x) \mathcal{L}x$)
We will also write $\int_A f(x) dx $, when $A$ is not an interval.


\subsection{Convergence of integrals}
 
\begin{center}
	\par\noindent\rule{200pt}{0.1pt}
\end{center}

\


\begin{center}		
For he strove in battles dire 

In unseen conflictions with shapes 

Bred from his forsaken wilderness. 

Of beast, bird, fish, serpent \& element 

Combustion, blast, vapour and cloud 

In infinite realms, areas unfold

Riemann's wings, summation told

Limits converge, precision's sway

Integration's dance, the mathematic way

Strong Law's grasp, a siren's call

Independent trials, destined to enthrall

\end{center}


\

\begin{center}
	\par\noindent\rule{200pt}{0.1pt}
\end{center}	
	
\	

\begin{definition}
	Let $X_1, X_2,...$ and be a sequence of random variables and $X$ be a random variable, all defined on the same probability space $(\Omega, \mathcal{F}, \Pp)$.
	We say that $X_1,X_2,...$ \textbf{converges to $X$ almost surely } and write $$\lim_{n \to \infty} X_n= X \quad \quad \textrm{almost surely} $$ if $$\Pp(\{ \omega \in \Omega \mid \lim_{n\to \infty }X_n(\omega)= X \})=1$$
Equivalently $$\Pp(\{ \omega \in \Omega \mid \lim_{n\to \infty }X_n(\omega) \neq  X \})=0$$
\end{definition}

A famous and important example is the \textbf{Strong Law of Large Numbers}.

\begin{definition}
	Let $f_1, f_2,...$ and be a sequence of real Borel-measurable functions and $f$ be another such function
	We say that $f_1,f_2,...$ \textbf{converges to $f$ almost everywhere } and write $$\lim_{n \to \infty} f_n= f \quad \quad \textrm{almost everywhere} $$ if  the set  $$\{ x \in \Rr \mid \lim_{n\to \infty }f_n(x) \neq  f(x) \}$$ has Lebesgue measure zero.
\end{definition}

\begin{theorem}[Monotone convergence]
	Let $X_1,X_2,...$ be a sequence of random variables convering almost surely to a random variable $X$. If $$0 \leq X_1 \leq X_2 \leq  \quad \quad \textrm{almost surely}$$
	then $$\lim_{n\to \infty} \Ee X_n = \Ee X. $$
	Let $f_1,f_2,...$ be a sequence of Borel-measurable functions on $\Rr$ converging almost everywhere to a function $f$. If $$0 \leq f_1 \leq f_2 \leq  \quad \quad \textrm{almost everywhere}$$
	then $$\lim_{n\to \infty}  \int_{-\infty}^{\infty} f_n(x)dx = \int_{-\infty}^{\infty} f(x)dx. $$
	
\end{theorem}


\begin{theorem}
	Let $X$ be a nonnegative random variable that takes countably many values $x_1,x_2,....$ Then $$\Ee X= \sum_{k=1}^{\infty}x_k \Pp({X=x_k})	 $$ 
\end{theorem}
 

\begin{theorem}
	[Dominated Convergence]. Let $X_1,X_2,...$ be a sequence of random variables convering almost surely to a random variable $X$. If there is a random variable $Y$ such that $\Ee Y < \infty $ and $|X_n|\leq Y$ almost surely for every $n$, then 
	  $$\lim_{n\to \infty} \Ee X_n = \Ee X. $$
	 
	Let $f_1,f_2,...$ be a sequence of Borel-measurable functions on $\Rr$ converging almost everywhere to a function $f$. If there is a function $g$ such that $$\int_{-\infty}^\infty g(x)dx < \infty $$ and $|f_n|<g$ almost everywhere for every $n$, then 
	 $$\lim_{n\to \infty}  \int_{-\infty}^{\infty} f_n(x)dx = \int_{-\infty}^{\infty} f(x)dx. $$
		
\end{theorem}
 
\subsection{Computation of Expectations}

\begin{center}
	\par\noindent\rule{200pt}{0.1pt}
\end{center}



\
	 
\begin{center}
Dark revolving in silent activity: \\
Unseen in tormenting passions; \\
An activity unknown and horrible; \\
A self-contemplating shadow, \\
In enormous labours occupied
\end{center}

\
\begin{center}
	\par\noindent\rule{200pt}{0.1pt}
\end{center}	
	
\	

Let $X$ be a random variable on a probability space ($\Omega, \mathcal{F}, \Pp$). We have defined the expectation of $X$ to be the (Lebesgue) integral:
$$\Ee X = \int_\Omega X(\omega)d \Pp (\omega), $$ naturally, this approch highlights the linearity of expectation and has a certain intuitive appel. 
But working on general abstract spaces is quite cancerous. So we will see a few theorems to relate a general $\Omega$ to $\Rr.$

\begin{theorem}
	Let $X$ be a random variable on a probability space $(\Omega, \mathcal{F}, \Pp)$ and let $g$ be a Borel-measurable function on $\Rr$. Then $$\Ee |g(X)|= \int_\Rr |g(x)|d \mu_X(x) $$ and if $\Ee |g(X)|< \infty $ then 
$$\Ee g(X)= \int_\Rr g(x)d \mu_X(x) $$

Let $X$ be a random variable on a probability space $(\Omega, \mathcal{F}, \Pp)$ and
let $g$ be a real Borel-measurable. Suppose that $X$ has a density $f,$ that is, $f$ is a nonnegative Borel-measurable function such that for every $B$ Borel subset of $\Rr$
$$\mu_X (B)= \int_B f(x)dx $$
Then $$\Ee |g(X)|= \int_{-\infty}^{\infty} |g(x)|f(x)dx $$ and if  $\Ee |g(X)|< \infty $ then 
$$\Ee g(X)=  \int_{-\infty}^{\infty} g(x)f(x)dx  $$
\end{theorem}
 



\subsection{Change of Measure} 
 
\begin{center}
	\par\noindent\rule{200pt}{0.1pt}
\end{center}

\

\begin{center}		
But Eternals beheld his vast forests. \\
Age on ages he lay, clos'd, unknown, \\
Brooding shut in the deep; all avoid \\
The petrific abominable chaos
\end{center}


\

\begin{center}
	\par\noindent\rule{200pt}{0.1pt}
\end{center}

\begin{theorem}
	Let $(\Omega, \mathcal{F}, \Pp)$ be a probability space and let $Z$ be an almost surely nonnegative random variable with $\Ee Z= 1$ For $A \in \mathcal{F}$ define $$\tilde{\Pp}(A)= \int_A Z(\omega)d \Pp(\omega). $$
Then $\tilde{\Pp}$ is a probability measure and if $X$ is a nonnegative random variable $$\tilde{E}X:= \int_\Omega X (\omega)d \tilde{\Pp}(\omega) = E[XZ] $$ If $Z$ is also surely (strictly positive), then $$\Ee Y = \tilde{E}\Bigg[\frac{Y}{Z}\Bigg]$$
for every nonnegative random variable $Y$.
\end{theorem}


 

\begin{center}
	\par\noindent\rule{200pt}{0.1pt}
\end{center}

\

\begin{center}		
His cold horrors silent, dark Urizen \\
Prepar'd; his ten thousands of thunders \\
Rang'd in gloom'd array stretch out across \\
The dread world. \& the rolling of wheels \\
As of swelling seas, sound in his clouds \\
In his hills of stor'd snows, in his mountains \\
Of hail \& ice; voices of terror, \\
Are heard, like thunders of autumn, \\
When the cloud blazes over the harvests
\end{center}

\

\begin{center}
	\par\noindent\rule{200pt}{0.1pt}
\end{center}


\end{document} 
